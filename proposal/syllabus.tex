\documentclass[12pt]{article}

\usepackage{fullpage}

% http://tex.stackexchange.com/questions/12279/outline-of-style-i-a-i-a-1
\usepackage{outlines}
\usepackage{enumitem}
\setenumerate[1]{label=\Roman*.}
\setenumerate[2]{label=\Alph*.}
\setenumerate[3]{label=\arabic*.}

\usepackage{url}
\usepackage[colorlinks=true]{hyperref}
\usepackage{parskip}

\begin{document}

\begin{center}
\Large
Computer Programming Course Outline\\
January 2017
\end{center}

\textbf{Learning goals:} 
\begin{itemize}
\item Students should leave this course with the
ability to write small programs in Java, such as the following:
\begin{itemize}
\item The ability to read data in from a CSV file and perform transformations
to the data or to visualize it.
\item The ability to write a small game (e.g., Tetris).
\item The ability to write a small communication service, such as a very
basic web browser or a chat client.
\end{itemize}
\item Students should leave this course with an understanding of the
inherent difficulty of computer programming and an appreciation for the
challenges of writing software in the real world.
\item Students will also learn more about modern computer systems (file/directory
structure, operating systems, networking) as we cover the more fundamental
topics.
\end{itemize}

The course will use the ACM Java Toolkit, as available at
\href{http://jtf.acm.org}{\texttt{jtf.acm.org}}. This Java library provides
students with a platform to easily create graphics using proper object-oriented
design, vastly simplifying the act of creating a graphical Java program. Students
will work in the \href{http://eclipse.org}{Eclipse} development environment,
a leading Java platform used by professionals.

There will be no textbook.

\textbf{Course outline:}
\begin{outline}[enumerate]
\setlength{\itemsep}{0pt}
\setlength{\parskip}{0pt}
\1 Basics
  \2 Working with Java programs
     \3 Textual input / punctuation
     \3 File formats
     \3 Running programs
     \3 Eclipse
  \2 Drawing shapes
     \3 Coordinate axes
     \3 Shape primitives
  \2 Mouse input
\1 Object orientation
  \2 Classes
  \2 Methods
  \2 Constructors
\1 Program control
  \2 Conditionals (\texttt{if})
  \2 Booleans
  \2 Animations
\1 Data
  \2 Numerical types
  \2 Variables
  \2 Fields
  \2 Parameters
\1 User interactivity
  \2 Keyboard input
  \2 Program state machines
  \2 Multimedia
     \3 Images
     \3 Sounds
     \3 Music
\1 Loops
  \2 Standard in/out
  \2 \texttt{while}
  \2 \texttt{for}
  \2 Primality checking
\1 Lists
  \2 \texttt{ArrayList}
  \2 List algorithms
     \3 min/max
     \3 reverse
     \3 delete
  \2 Strings
\1 Files
  \2 Basic file input/output
  \2 Parsing/tokenizing
  \2 Data manipulation
  \2 Data visualization
\1 Recursion
  \2 Recursion on numbers
  \2 Recursion on strings
  \2 Recursion on lists
\1 Inheritance
  \2 Interfaces
  \2 Abstract methods
  \2 Abstract classes
  \2 Overriding
  \2 Class-based polymorphism
\end{outline}

Students will be assessed on weekly programming assignments, several quizzes
and exams, and several longer projects. Longer projects will include a
\href{http://www.ponggame.org/}{Pong}-like game, a real-world data
manipulation \& visualization (e.g., using data downloaded from
\href{http://data.gov}{\texttt{data.gov}}), and a final project of the
student's choosing, which will serve as the final assessment.
A student's grade will be based primarily on their out-of-class written work.
Quizzes and exams will serve only to ensure that the student herself has
learned the material and has the basics readily accessible.

\end{document}
